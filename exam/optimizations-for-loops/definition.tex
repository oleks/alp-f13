\tableofcontents

\section{Motivation}

\begin{itemize}

\item Programs tend to spend most of their time in loops.

\item A loop is a program unit (body) that is repeatedly executed, having a
single entry point:

\begin{itemize}

\item Any path from the start of a function to the body of a loop goes through
the entry point of the loop.

\item From any path in the body of the loop, there is a path to the entry point
of the loop.

\end{itemize}

\end{itemize}

\section{Code Hoisting}

\begin{itemize}

\item Loop-invariant computations.

\item Lifting such computations outside the loop is \emph{code hoisting}.

\item Available assignment did something similar, but if computed with a loop,
common subexpression elimination has no previous assignment available to make
this conclusion.

\item Moving or copying the instruction to a place before the loop also causes
the value to be computed even if the loop is a 0-trip.

\item In addition to loop invariant, the instruction must also be valid, even
if the condition guarding the loop is not satisified, this could be important
wrt.  e.g. memory access.

\item One viable solution is to unroll the loop once (while -> if-while).

\item Assuming that we subsequently perform common subexpression elimination,
this is enough to perform code hoisting.

\item This is easy enough at the source code level, but can be a little
challenging at the level of the intermediate code, care must be taken of jumps.

\end{itemize}

\section{Memory Prefetching}

\begin{itemize}

\item Modern processors have memory prefetch instructions. They do not fail if
the adress is invalid, but load an address into memory so that when the time
comes to use it, the processor need not wait around, but simply fetch it from
the cache.

\item We can transform a loop to prefetch a bank of memory (say 4 words), while
we're working on the previous bank. The numer 4 may vary depending on the size
of the loop.

\item This also require a test to ensure that the next bank is only loaded once
on entry to a new bank, this test can be mitigated for by loop unrolling, but
be vary of code explosion.

\end{itemize}

\section{Incrementalisation}

\begin{itemize}

\item General idea: replace $f(x + \delta)$ using $f(x)$ and $\delta$.

\item For instance $(x+1)^2=x^2+2x+1$, so it can be done by computing $v=x^2$
and $v+x+x+1$ or even $v+(x<<1)+1$.

\item Also known as \emph{reduction in strength}, as it reduces the complexity
of an operation.

\item We must take care to introduce arithemtic overflow in much the same
fashion as the original code.

\item If operating on floating point, rounding errors are also worth
consideration.

\item Another example is array index calculation, so if we know the address
$a[i][j]$, we can quickly compute the address of $a[i+1][j]$, if we know the
first size dimension (in bytes) of the nested array.

\item Rules:

\begin{itemize}

\item Code in consideration (clear definition, typically a loop or the body of
a function.

\item Loop counter is a variable of either the form $i:=c$ or $i:=i+k$, where
neither $c$ or $k$ are changed anywhere in the code under consideration.

\item A variable derived from a loop counter is a variable that has a single
assignment in the code in consideration of the form $x:=y+k$ or $x:=y*k$, where
$k$ is not changed in the code in consideration, and $y$ is either a loop
counter or another variable derived from a loop counter. $x$ must be dead on
entry to the code under consideration, and dead on exit. No assignment to $i$
can happen between the declaration of the variale and its use.

\end{itemize}

\item A variable $x$ derived from a loop counter $i$ will have the form
$i*p+q$, where $p$ and $q$ are not changed in code under consideration.

\begin{align}
x:=i+k & x = i*1 + k \\
x:=i*k & x = i*k + 0 \\
x:=y+k \wedge y=i*p+q & x = i*p+(q+k) \\
x:=y*k \wedge y=i*p+q & x = i*(pk)+(qk)
\end{align}

\item $q+k$, $pk$ and $qk$ can be precomputed, as their values do not change in
the code under consideration.

\item if $x:=i*p+q$ then if $i:=c$, replace it with $x:=(pc)+q$, if $i:=i+k$,
replace it with $x:=x+(pk)$. ($pk$ and $pc$ can be precomputed). If $i$ is live
on entry, add $x:=i*p+q$ before init.

\item if $k$ doesn't change in the code under consideration, we can also
replace a test $i<k$ with $x<(pk)+q$ if $p>0$ and $x>(pk)+q$ if $p<0$. If $p=0$
or the sign is unknown, we can't make the replacement.

\item similar replacements for other relations. ($pk+q$ can also be precombputed).

\end{itemize}
