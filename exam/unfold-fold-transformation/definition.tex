\tableofcontents

\begin{description}[\setleftmargin{20pt}\setlabelstyle{\bf}\breaklabel]

\item [lol]

\end{description}

\section{Motivation}

\begin{itemize}

\item Generalization of \emph{partial evaluation}, \emph{deforestation},
\emph{tupling}, \emph{supercompilation}, etc.

\end{itemize}

\section{Language}

\begin{itemize}

\item Subset of Haskell.

\item A function is a series of rules.

\item A a rule has the form $f\ p = e$, where $f$ is an fid, $p$ is a pattern,
and $e$ is an expression.

\item Linear patterns: no variable can occur more than once in a pattern.

\item Rules non-overlapping, i.e. for any $f\ p_1 = e_1$ and $f\ p_2 = e_2$,
$p_1$ and $p_2$ cannot be unified.

\item Assume type inference.

\item Lets are expanded: \texttt{f x = let y = g x in h(x, y, y)} becomes
\texttt{f x = l (g x, x); l (y, x) = h(x, y, y)}.

\item A value is is a an expression built from constructors and tuples.

\item We run a program by calling a function with values as arguments, and
evaluating the call to a value.

\item This may fail due to nontermination or inexhaustive pattern matching.

\end{itemize}

\section{Evaluation}

\begin{itemize}

\item Evaluation is a series of function applications.

\item The order of evaluation does not matter, it only affects termination, but
not the result.

\item A \emph{substitution} is a set of bindings of variables to expressions.

\item A binding is written $x\setminus e$, where $x$ is the variable and $e$ is
the expression.

\item A substitution is a list of such bindings s.t. $x_i = x_j \rightarrow i =
j$ and $x_i$ does not occur in $e_i$ or $e_j$. 

\item This means that a substitution is idempotent, so the order of evaluation
indeed does not matter.

\item We combine two non-overlapping substitutions using the operator $+$.

\item We apply a substitution $\Theta$ to $e$ ($e\Theta$) as shown in
\referToFigure{substitution}.

\includeFigure[scale=0.7]{substitution}{Substitution.}

\item We obtain substitutions by matching a pattern to an expression, as shown
in \referToFigure{pattern-matching}.

\includeFigure[scale=0.7]{pattern-matching}{Pattern matching.}

\item Assume no variable occurs more than once in $p$, and no variable in $p$
occurs in $e$.

\item If we have a call $f e1$ and a rule $f\ p = e2$ s.t. $p\lhd
e=\Theta\neq\mathbf{fail}$, then $f e1$ is replaced by $e2\Theta$.

\item The rule may additionally be renamed to ensure that it shares no
variables with $e1$, as these variables are bound anew, and are seperate from
the other bindings.

\item As function rules do not overlap, at most one ruls is matched, but none
could be matched as well.

\item We repeat function application until no calls remain in the expression.

\end{itemize}

\section{Transformation steps}

\subsection{Unfolding}

\begin{itemize}

\item This is function application.

\item Unfolding should only happen for calls that have exactly one matching
rule in the function definition.

\end{itemize}

\subsection{Definition}

\begin{itemize}

\item The addition of a new (uniquely named) function to the program.

\end{itemize}

\subsection{Extraction}

\begin{itemize}

\item Replacing one ore more subexpressions in an expression by a variable.

\item This requires a let binding, which is short for a function call.

\end{itemize}

\subsection{Folding}

\begin{itemize}

\item Function application in reverse: replace one or more subexpressions by a
function call.

\item We keep the original definition around, so as to fold a part of a
transformed rule with the original; this allows us to generate new recursive
functions.

\item One folding step should be preceeded by at least one unfolding step to
avoid circular definitions.

\end{itemize}

\subsection{Special casing}

\begin{itemize}

\item Replace a pattern variable with a destructor, replace uses of the
variable with adequate constructors in the rule expression.

\item Add rules for other constructors to ensure same coverage.

\end{itemize}

\section{Transforming a program}

\begin{itemize}

\item Apply a series of the above transformations, taking care not to change
the meaning of the program.

\item We often add function definitions iteratively.

\item Consider the program

\begin{verbatim}
data List a = Nil | Cons (a, List a)

append (Nil, ys) = ys
append (Cons (x, xs), ys) = Cons (x, append (xs, ys))
\end{verbatim}

\item Add the function definition

\begin{verbatim}
append3 (xs, ys, zs) = append (append (xs, ys), zs)
\end{verbatim}

\item Note that it would be more efficient to perform the inner append call in
the second argument as it is the first list that gets iteratively destructed by
\texttt{append}.

\item Perform the following in order:

\begin{enumerate}

\item Specal-case:

\begin{verbatim}
append3 (Nil, ys, zs) = append (append (Nil, ys), zs)
append3 (Cons (q, qs), ys, zs) =
  append (append (Cons (q, qs), ys), zs)
\end{verbatim}

\item Unfold inner \texttt{append}:

\begin{verbatim}
append3 (Nil, ys, zs) = append (ys, zs)
append3 (Cons (q, qs), ys, zs) =
  append (Cons (q, append (qs, ys), zs)
\end{verbatim}

\item Unfold outer \texttt{append}:

\begin{verbatim}
append3 (Nil, ys, zs) = append (ys, zs)
append3 (Cons (q, qs), ys, zs) =
  Cons (q, append (append (qs, ys), zs)
\end{verbatim}

\item Fold (using the \emph{original} definition of \texttt{append3}):

\begin{verbatim}
append3 (Nil, ys, zs) = append (ys, zs)
append3 (Cons (q, qs), ys, zs) = Cons (q, append3(qs, ys, zs))
\end{verbatim}

\end{enumerate}

\item This is more efficient.

\end{itemize}

\section{Strategies for Unfold-fold Transformations}

\begin{itemize}

\item Look for \emph{configurations}, and use \emph{extraction} to make
definitions for these configurations.

\begin{description}[\setleftmargin{20pt}\setlabelstyle{\bf}\breaklabel]

\item [Partial evaluation] A configuration is a function call with one or more
arguments; we reduce the number of arguments by eliminating constant
parameters.

\item [Deforestation] A configuration is a series of nested function calls; we
reduce the creation intermediate data (or stack space), e.g. \texttt{append3}.

\item [Tupling] A configuration is a series of nonnested function calls; we
reduce the traversal of the same data structure.

\item [Super(vised)compilation] Any expression that contains multiple calls; we
make all of the above optimizations, and prove certain properties about
functions: compose $f$ with some property function $q$, if the result reduces
to $\top$, then the property always holds for $f$.

\end{description}

\item When do we apply unfolding and specal-casing? (Rules of thumb):

\begin{enumerate}

\item We look for subexpressions that can be unfolded int the defined/extracted
rules. If none exist, e.g. due to inexhaustive pattern matching, we
special-case a variable $x$ in the expression.

\item Any call that is defined by multiple rules, but only one rule matches the
call, is unfolded. Additionally, no parameter expression containing a function
call may be duplicated or eliminated during unfolding. TODO: hmm..

\item When no further unfolding can be applied, the defined/extracted rules are
folded with their (most recent) originals.

\item If the above steps produce a new definition, repeat. 

\end{enumerate}

\item Example \texttt{append3}, but no new definitions were generated.

\item Infinite sequences possible, so we keep track of the unfoldings,
considering their implications on size and complexity.

\item There can be a infinite number of configurations, this is detected as
above, and fixed by generalisation, i.e. considering subconfigurations.

\item We define a partial order on the values that can be generated, $\unlhd$,
e.g. their cardinality.

\item A sequence that decreases the size of a tree cannot possible be infinite.

\item This still leaves many trees as possibly infinite, so instead we use
\emph{homomorphic embedding} as shown in \referToFigure{homomorphic-embedding}.

\includeFigure[scale=0.7]{homomorphic-embedding}{Homomorphic embedding.}

\item That is, $s\unlhd t$ is $s$ can be obtained from $t$ by e.g. deleting
constructors or function calls (increased complexity), replacing the tuple by
one of its constituents, or defining variables.

\end{itemize}

\section{Extensions and Issues}

\begin{itemize}

\item What about additional language features?

\end{itemize}

\subsection{Numbers}

\begin{itemize}

\item Number constants as constructors and arithmetic operations as functions.

\item Include arithmetic simplification.

\item $m\unlhd n$ iff $m\leq n$ for natural numbers; negatives and fractions
are tough however.

\item Another option is piano numbers, perhaps with binary representation.

\end{itemize}

\subsection{Higher-order functions}

\begin{itemize}

\item Straight-forward.

\end{itemize}

\subsection{Code Duplication and Elimination}

\begin{itemize}

\item When unfolding, parameters may be duplicated, this increases computation
time.

\item These may further be eliminated by folding, but this can alter program
behaviour.

\item A non-terminating program may terminate, or an erroneous program may
produce a non-error result.

\item We restrict unfolding to not duplicate or eliminate function calls.

\end{itemize}

\subsection{Side effects}

\begin{itemize}

\item The transformations can alter the order of evaluation.

\item If side effects are allowed, this may cause an altered program behaviour.

\item Parameters may therefore not produce or depend on side effects in
general.

\item Monadic operators ensure the order of side effects regardless of the
order of evaluation.

\end{itemize}

\subsection{Exceptions}

\begin{itemize}

\item Similar to side effects.

\end{itemize}

\subsection{Accumulating Parameters}

\begin{itemize}

\item Don't typically lend themselves to unfold-fold transformations.

\item It is not possible (using the above techniques) to optimize two nested
calls to list reverse with an accumulating parameter.

\end{itemize}

\section{Examples of Transformations}

\begin{itemize}

\item \texttt{append3} was an example of deforestation.

\end{itemize}

\subsection{Tupling}

\begin{itemize}

\item Combine several calls that recurse over the same structure.

\item Can sometimes reduce execution time from exponential to linear.

\item Consider the (exponential-time) Fibonacci function:

\begin{verbatim}
data Int = Z | S Int

add (Z, y) = y
add (S x, y) = S (add (x, y))

fib Z = Z
fib (S Z) = (S Z)
fib (S (S x)) = add (fib x, fib (S x))
\end{verbatim}

\item We add \texttt{fib2}:

\begin{verbatim}
fib (S (S x)) = add (fib2 x)
fib2 x = (fib x, fib (S x))
\end{verbatim}

\item Specah-case:

\begin{verbatim}
fib2 Z = (fib Z, fib (S Z))
fib2 (S y) = (fib (S y), fib (S (S y)))
\end{verbatim}

\item Unfold all calls to \texttt{fib} for which only one rule matches: 

\begin{verbatim}
fib2 Z = (Z, S Z)
fib2 (S y) = (fib (S y), add (fib y, fib (S y))
\end{verbatim}

\item Extract in the second rule:

\begin{verbatim}
fib2 (S y) =
  let (p,q) = (fib y, fib (S y))
  in (q, add (p, q))
\end{verbatim}

\item Fold the second rule with the original definition:

\begin{verbatim}
fib2 (S y) = let (p, q) = fib2 y in (q, add (p, q))
\end{verbatim}

\item This has a linear number of additions.

\end{itemize}

\subsection{Partial evaluation}

\begin{itemize}

\item Specialize programs by eliminating the constant parameters.

\item Configurations are function calls with variables and values.

\item Consider

\begin{verbatim}
data Int = Z | S Int
ackermann (Z, n) = S n
ackermann (S m, Z) = ackermann (m, S Z)
ackermann (S m, S n) = ackermann (m, ackermann (S m, n))
\end{verbatim}

\item We want to specialize by fixing the first parameter to $3$:

\begin{verbatim}
ackermann3 n = ackermann (S (S (S Z)), n)
\end{verbatim}

\item Special case on \texttt{n}:

\begin{verbatim}
ackermann3 Z = ackermann (S (S (S Z)), Z)
ackermann3 (S p) = ackermann (S (S (S Z)), S p)
\end{verbatim}

\item Unfold both rules:

\begin{verbatim}
ackermann3 Z = ackermann (S (S Z), S Z)
ackermann3 (S p) = ackermann (S (S Z), ackermann (S (S (S Z)) p))
\end{verbatim}

\item Unfold first rule a couple more times:

\begin{verbatim}
ackermann3 Z = S (S (S (S (S Z))))
ackermann3 (S p) = ackermann (S (S Z), ackermann (S (S (S Z)), p))
\end{verbatim} 

\item Fold the second rule with the original definition:

\begin{verbatim}
ackermann3 Z = S (S (S (S (S Z))))
ackermann3 (S p) = ackermann (S (S Z), ackermann3 p)
\end{verbatim}

\item We define \texttt{ackermann2} and fold with it:

\begin{verbatim}
ackermann3 Z = S (S (S (S (S Z))))
ackermann3 (S p) = ackermann2 (ackermann3 p)
ackermann2 x = ackermann (S (S Z), x)
\end{verbatim}

\item Special-case and unfold a couple times:

\begin{verbatim}
ackermann2 Z = S (S (S Z)
ackermann2 (S q) = ackermann (S Z, ackermann (S (S Z), q))
\end{verbatim}

\item Define \texttt{ackermann1} and fold with it:

\begin{verbatim}
ackermann2 Z = S (S (S Z)
ackermann2 (S q) = ackermann1 (ackermann2 q)
ackermann1 y = ackermann (S Z, y)
\end{verbatim} 

\item Special-case and unfold a couple times:

\begin{verbatim}
ackermann1 Z = S (S Z)
ackermann1 (S r) = S (ackermann1 r)
\end{verbatim}

\item If the original function (\texttt{ackermann}) is not used, it can be
removed from the program text entirely.

\end{itemize}

\subsection{Supercompilation}

\begin{itemize}

\item Supercompilation is a wise combination of deforestation, tupling and
partial evaluation.

\item In addition to property proving, we can prove program equivalence, if
they optimize to the same program.

\item We give an example of proof by supercompilation.

\item Consider 

\begin{verbatim}
data Int = Z | S Int
add (Z, y) = y
add (S x, y) = S (add (x, y))
\end{verbatim}

\item To prove by supercompilation we add the following definitions
(associativity, $0+y=y+0$):

\begin{verbatim}
equal (Z, Z) = True
equal (Z, S y) = False
equal (S x, Z) = False
equal (S x, S y) = equal (x, y)
test x = equal (x, add (x, Z))
\end{verbatim}

\item Special-case:

\begin{verbatim}
test Z = equal (Z, add (Z, Z))
test (S p) = equal (S p, add (S p, Z))
\end{verbatim}

\item Repeatedly unfold:

\begin{verbatim}
test Z = True
test (S p) = equal (p, add (p, Z))
\end{verbatim}

\item Fold with the original \texttt{test}:

\begin{verbatim}
test Z = True
test (S p) = test p
\end{verbatim}

\item It is clear that this, if it terminates will yield \texttt{True}. The
value is monotonically decreasing so the test terminates.

\end{itemize}

\subsection{Generalisation}

\begin{itemize}

\item None of the above required it.

\end{itemize}
