\tableofcontents

\begin{description}[\setleftmargin{20pt}\setlabelstyle{\bf}\breaklabel]

\item [lol]

\end{description}

\section{Motivation}

\begin{itemize}

\item The functions in IM are flatly scoped (there is only one global scope);
function cannot be stored and passed as values.

\item Introduce indirect calls (ICALL), where the function name is stored in a
variable rather than given as a constant identifier.

\item In machine language we'd just store a constant address in a register
(memory).

\item If a processor has no branch and link, we can implement this by storing
the program counter, and storing the jumped to address in the program counter.

\end{itemize}

\section{Summary of translation}

\begin{itemize}

\item Functions use register to store variables, if the set of registers is
insufficient, we spill onto the stack.

\item We split the processor registers into \emph{caller}-saves and
\emph{callee}-saves.

\item A function body is transformed to have a \emph{prelude}.

\begin{itemize}

\item Function label.

\item Allocate space on the stack of the local variables.

\item Store callee-saves register, and perhaps return address.

\item Transfer parameter to local variables.

\end{itemize}

\item A return instruction is transformed into an \emph{epilogue}

\begin{itemize}

\item Code for transferring the return value from local variables to
appropriate registers or stack.

\item Restore callee saves registers, and perhaps return address.

\item Free up space used for local variables.

\item Return to the return address.

\end{itemize}

\item A function call is transformed into a \emph{call sequence}.

\begin{itemize}

\item Save \emph{live} variables stored in the caller-savers registers.

\item Code for transferring the parameters from local variables to the
registers or stack locations used to pass parameters to the callee.

\item Code for storing the return address and jumping to the callee.

\item

\item Code for transferring the return value from the register or stack to
local variables.

\item Restore local variables saved in caller-saves registers.

\end{itemize}

\item Each type of processor typically has its own procedure call standard.

\end{itemize}

\section{C-style functions}

\begin{itemize}

\item All function declarations occur at the top level.

\end{itemize}

\section{Nested function declarations}

\begin{itemize}

\item Pascal.

\item f has g has h, f can call g, but not h, h can call both g and f; static
scoping.

\item The above prologue, epilogue, call sequence trio don't fit this paradigm;
however this works seamlessly at the typechecker level.

\item We can rewrite the nested function declarations to top-level function
declarations.

\item Recall that the memory is accessible to all functions.

\item Two choices: either pass the variables as extra parameters (becomes
cumbersome if we also want to modify the variables in the outer scope), or
store the variables in globally accessible memory, and pass pointers to them.

\item Allocation can happen on the stack, but should be symmetric, on return
the local variable corresponding to the memory cell must reloaded.

\item Allocation of passed variables can happen in blocks, and the pointers can
be offset from the beginnings of these blocks.

\item A block of memory that holds all the memory-resident variables of a
function is called an \emph{environment}.

\item A pointer to the environment is called a \emph{static link}, as the link
to the return address is often called the \emph{dynamic link}.

\item The environemnts from outer scopes form a linked list connected by static
links.

\item Accessing elements higher up thus requires multiple (possibly distant)
memory accesses.

\end{itemize}

\section{Functions as Parameters}

\begin{itemize}

\item Unlike in C, in Pascal it is not sufficient to pass the address of a
function to pass a function parameter, we may need to include the environment
as well.

\item A functional parameter is a pair of a function address and a static link,
called a \emph{closure} or a \emph{thunk}.

\item \emph{Closure conversion}: translation to use explicit closures.

\item In Pascal, functions are \emph{not} first-class citizens, as functional
values cannot be returned and stored (only passed as parameter), the
environment and closure can be stored on the stack, i.e. there is no need for
the closure once the callee returns.

\end{itemize}

\section{First-class functions}

\begin{itemize}

\item First-class functions disallow the allocation of closures on the stack,
and disallow the allocation of variables stored in the closure on the stack, we
\emph{must use the heap}, as with first-class functions closures can outlive
the lifetime of the enclosing function.

\item The lifetime of a closure can be hard to predict, and therefore languages
that allow functions as first-class citizens typically use automatic memory
management.

\item However, such languages typically have other properties that can make the
lifetime of a closure predictible in certain cases.

\end{itemize}

\section{Dynamic scoping}

\begin{itemize}

\item Every variable has a global stack, as the runtime enters a function body,
the value for the variable is pushed on the stack (perhaps altered during
function execution), as the runtime exits a function body, the variable stack
is popped, and the previous value of the variable is restored.

\item There is no need for environments and closures (I guess this is why good
old Lisp had dynamic scoping, successors of Lisp use static scoping).

\item This has pitfalls, if you pass a reference to a value, that reference may
no longer point to the same value on dereference. Likewise, referencing a
variable in the outer scope, may well reference a completely different
variable. This disallows e.g. continuations.

\end{itemize}

\section{Functional programming languages}

\begin{itemize}

\item A functional PL is a PL where the primary control structure is a function
call.

\item Typically functions are first-class citizens, but assignment is
restricted through an adequate type system.

\item A pure functional PL disallows destructive assignment.

\item Impure functional languages

\begin{itemize}

\item If there are no limitations on assignment, we cannot make optimization to
the above closures. 

\item However, in e.g. SML we cannot assign to variables after creation, but we
can assign to references after creation; so variables can be freely copied
without fear of change.

\item More complex structures (e.g. tuples) are still allocated on the heap
however.

\end{itemize}

\item Pure functional languages

\begin{itemize}

\item No assignments, no side effects.

\item Two calls with the same arguments will lead to the same value, leading to
a reduced number of function calls. 

\item We can also rearrange the order of evaluation if this is deemed beneficial.

\end{itemize}

\item Lazy functional languages

\begin{itemize}

\item Postpone evaluation of parameters until its value is needed.

\item Call-by-name: reevaluate the parameter \emph{every} time it is needed.

\item Algol 60 used call-by-name for parameters (but not initialisation), it
was impure so different reevaluation lead to different values, this exploited.

\item Call-by-name is implemented by transforming all parameters into
parameterless functions, and evaluating the thunk when needed.

\item If the parameter is itself a variable, no need to wrap it one more time
(eta reduction in lambda calculus).

\item Call-by-need: reduces the number of reevaluations, by replacing the
variable with an evaluated closure once it has been computed once, this can be
added automatically as a wrapper to function in the closure.
\end{itemize}

\item Strictness analysis is used to reduce some of the overhead: if the value
is guaranteed to be used in an expression, it is evaluated from the get go
(call-by-value). See \referToFigure{strictness}.

\includeFigure[scale=0.5]{strictness}{Strictness}

\begin{itemize}

\item We say that $x$ is strict in $f$, if $x$ is definitely needed to evaluate
$f$.

\end{itemize}

\end{itemize}

\section{Optimisations for Function Calls}

\begin{itemize}

\item Function calls impose overhead.

\item Inlining

\begin{itemize}

\item Completely eliminate the overhead by inlining the body instead of the
call.

\item Beware of variable capture, a calling function $f$ uses a local $x$,
while the called function $g$ uses a global $x$. By scoping rules, this will
make $x$ in $g$ refer to the local variable in $f$ if we perform trivial
inlining.

\item Instead, we define a function $g'$, rename all variables in $f$ and $g'$,
inline $g'$, and remove it from the source text.

\item For languages, where function definitions can be nested, we need to
rename the entire declaration hierarchy.

\item One must take care, as inlining the same function multiple times may lead
to larger programs, larger function bodies may lead to more spilling.

\item Typically, only small function are inlined.

\item Another problem is small recursive functions, inlining them in the caller
is generally wasteful, but inlining them in themselves can reduce the overhead
considerably.

\end{itemize}

\item Tail-call optimization

\begin{itemize}

\item The value returned is the result of a function call. (Between the call
and the return there may be unconditional jumps.)

\item Typically can only be done at the IL or assembly level.

\item No live variables need to be stored in the caller-saves registers, only
the return adress is interesting.

\item No need to read the return value into a local variable, only to store it
back in the return register.

\item Free the stack and restore callee saves registers before the final
function call, but beware of where the return address is saved, perhaps it
cannot be restored quite yet.

\item We can then promote the return address of the current function as the
return address for the called function.

\item These changes imply that the return address is either at the top of the
stack, or in the link register as we are ready to perform the jump, the jump is
therefore a bare bones jump.

\item Tail-call optimization is important in language without loop control
structures. FPL's typically fill caller-saves registers first, this enables the
bare bones jump implementation as discussed above. 

\end{itemize}

\item Tail-recursion optimization

\begin{itemize}

\item Further optimization is applicable if the calle is the same function as
the current.

\item We can omit the callee saves register in the recursive calls, as they are
already adequately saved.

\item The same goes for allocating stack space for local variables.

\item And hence the parts of the prologue and epilogue moving the arguments
back and forth between adequate registers and the local variables.

\item Tail-recursion optimization (unlike tail-call optimization), can be
performed at the IL level.

\end{itemize}

\end{itemize}

\section{Exceptions}

\begin{itemize}

\item Exceptions allow a function to return across multiple calls to an
exception handler.

\item Two constructs in the language: raise and handle.

\item A raise passes control to the most recently active handler.

\item A handler is active if it has been entered, but not exited.

\item Tagged return

\begin{itemize}

\item In addition to the result, we return a value indicating whether all is
well or an exception was raised.

\item The indicator can be a dedicated register or a dedicated memory location.

\item Considerable overhead as this has to be added to the call sequence of a
function call, it even touches tail-call and tail-recursion optimization.

\item This feature affects program that do not use this feature --- not so
nice.

\item Static analysis can allow to detect whether a function at can raise an
exception or not, and in the latter case the check can be eliminated.

\item However, even adding the check to only the functions that \emph{might}
raise an exception is not ideal.

\item An alternative is to have an exception handler address in addition to the
return address; if a callee has no exception handler, one is added.

\end{itemize}

\item Stack unwinding

\begin{itemize}

\item Attempt to move the cost only to those programs that raise exceptions.

\item For each return adrres on the stack of stack frames, an address for the
local handler in the callee (if any) is added.

\item When an exception occurs, the stack frames are observed in reverse order,
and the stack restored accordingly.

\item There is (litle) no cost to exceptions if they don't occur, but if they
do, it depends on the depth of the stack.

\end{itemize}

\item Handler stack

\begin{itemize}

\item On entry to an exception handler, the live variables are stored, along
with the address of the handled on the dedicated handler stack.

\item When leaving the exception handler, the handler is popped off the handler
stack.

\item When an exception is encountered, the exception handler is directly
available, and all the live variables before entering the handler can be
restored.

\item Whether this method is better than stack unwinding depends on the
frequency of handlers contra frequency of exceptions occuring. When
implementing Prolog, a handler stack is to be preferred.

\end{itemize}

\end{itemize}
